\chapter{Diskussion}
\section{Diskussion}
Det har varit väldigt lärorikt att jobba med det här projektet och hela kursen i sig har varit väldigt lärorik. Vi förstår mycket bättre nu hur kraftfullt XML är som ett verktyg för att beskriva/förmedla data på ett standardmässigt sätt och hur det kan användas ihop med andra programmeringsspråk för att underlätta kommunikation. Med hjälp av XSL tranformationer blir det möjligt att transformera XML till olika publiceringskanaler som t.ex. XHTML för webbsidor eller till Word/Excel (även PDF med XSL-FO). Vi använde oss av RSS feeds för att få reda på vilka de senaste avsnitten är för en viss serie och fick därmed en hel del övning med hur man kan parsa XML med PHP och det blev ännu tydligare hur det kan underlätta att ha en standard som XML som i fallet med RSS feeds för att bygga applikationer.

Det finns många fördelar med att använda XML för internettjänster för flerkanalspublicering. I vårt fall med vårt projekt var det kanske lite för krystat. Vi har egentligen ingen riktig användning av flerkanalspublicering så som vi har gjort det med Excel. Det är svårt att se vad en användare skulle göra med ett Excel dokument med information om alla serier han/hon följer. I fallet med admin skulle, å andra sidan, ett Excel dokument kunna användas i ett administrativt syfte för statistik eller något liknande. 

En nackdel vi märkte var att det blev svårare/mindre flexibelt att designa HTML sidorna för tjänsten med XSL då vi var tvugna att matcha mot de taggar vi definierat för XML. Detta gjorde det svårare att placera HTML element var som helst i dokumentet. Vi kunde heller inte använda PHP script direkt i den html-genererade filen vilket ledde till att vi fick ta några omvägar med t.ex. JavaScript o.dy. 

En nackdel med hur vi har byggt tjänsten är att databasen inte uppdateras dynamiskt med nya Tv-serier utan måste ske med manuellt arbete. Man måste först hämta namn och ID (för RSS feeds) för serier från showrss, göra querys till OMDb API:et, parsa responsen och lägga till i databasen. Utöver detta måste vi själva ladda ner alla posters då URLerna som vi erhåller från OMDb är från imdb och som inte tillåter hotlinking. 
